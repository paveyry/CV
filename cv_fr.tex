\documentclass[11pt,a4paper]{moderncv}
\moderncvtheme[red]{classic}                
\usepackage[utf8]{inputenc}
\usepackage[top=1.0cm, bottom=1.0cm, left=2cm, right=2cm]{geometry}
\usepackage{./moderntimeline}

\setlength{\hintscolumnwidth}{2.7cm}

\firstname{Pierre-Alexandre}
\familyname{VEYRY}
\title{Étudiant en ingénierie informatique}             
\address{Les jardins de Paste}{07000 PRIVAS}
\mobile{06.48.79.87.79}                    
\email{veyry\_p@epita.fr}
\homepage{pa.veyry.fr}
\extrainfo{20 ans -- Permis B}           

\begin{document}

\tlmaxdates{2011}{2015}
\tlwidth{0.8ex}
\tltext{\tiny}

\maketitle

\section{Cursus}

\tldatecventry{2012}{Baccalauréat Scientifique}{Mention Très-Bien}{}{}{}

\tlcventry{2012}{0}{École d'ingénieurs en Informatique}{EPITA}{Le Kremlin-Bicêtre}{}{}

\tldatecventry{2014}{Échange international d'un semestre}{Staffordshire University}{Royaume-Uni}{}{}


\section{Expériences}
	\subsection{Projets scolaires}
	\tlcventry{2012}{2013}{Jeu vidéo}{Metastruggle}{Un Super-Smash-Bros-like en 3D développé en C\# avec XNA}{}{\url{http://en.metastruggle.epita.it}}
	\tldatecventry{2013}{OCR}{Un logiciel de reconnaissance de caractères en OCaml}{}{}{}
	\tldatecventry{2014}{HTTPd}{Un serveur HTTP en C}{}{}{}
	\tldatecventry{2014}{Malloc}{Une implémentation de Malloc du type best-fit}{}{}{}
	\tldatecventry{2014}{42sh}{Une imitation de bash --posix en C}{}{}{}
	\tldatecventry{2014}{FlatSet}{Une implémentation de std::set basée sur std::vector en C++}{}{}{}


	\subsection{Emplois et stages}
	
	\tlcventry{2014}{0}{Assistant enseignant (ASM)}{EPITA}{Villejuif}{France}
	{Enseignement du C et d'UNIX aux élèves de deuxième année de l'école EPITA}
	\tldatecventry{2014}{Développeur Web}{SYDER}{Développement de différentes applications Web en HTML, CSS, Javascript et PHP}{}{}

\section{Compétences techniques}
	\subsection{Langages de programmation}
        \cvcomputer{Compétent}{C, C++, PHP, JS, SQL, \LaTeX}{Frameworks}{Bootstrap}
        \cvcomputer{Expérimenté}{C\#, OCaml, HTML, CSS}{Frameworks}{Laravel, SFML, .NET, XNA}
	\cvcomputer{Débutant}{Python, Java}{}{}
	\subsection{Divers}
        \cvitem{Systèmes}{GNU/Linux, Mac OS X, Windows, BSD}{}{}{}{}
        \cvitem{Logiciels}{Blender, Photoshop, Gimp, Cubase}{}{}{}{}
	
	\section{Langues}
	\cvlanguage{Français}{Langue natale}{}
        \cvlanguage{Anglais}{Avancé (TOEIC:980/990)}{}
	\cvlanguage{Allemand}{Notions}{}

\section{Centres d'intérêt}

\cvitem{Arts}{Piano (Pratiqué pendant 11 ans), Musique, Cinéma, Littérature, Histoire de l'Art}{}{}{}{}

\cvitem{Sports}{Triathlon, Canoë-Kayak}{}{}{}{}


\end{document}
